\begin{paragraph}
  MarTeX modules are easy to develop. An empty module looks like this: 
\end{paragraph}

\begin{code}{php}
\<?php
// Add this module to the MarTeX namespace
namespace MarTeX;

// Include the base class and module tools
include __DIR__.'/module.php';

// Define the class and extend the MarTeXmodule
class Helloworld extends MarTeXmodule \{

\}

?\>   
\end{code}
That's all there is to it! You now have a module that does nothing...\\
Some remarks before we get to adding functionality to the module:
\begin{itemize}
 \item{The filename of the module should be all lowercase.}
 \item{The classname of the module should be the filename with the first letter capitalized.}
 \item{The class must be defined in the MarTeX namespace.}
\end{itemize}
That means we should save the example above as 'helloworld.php'.\\
Now for adding functionality. There are three types of ways your module can add functionality: commands, environments and previronments. 
\subsubsection{commands}
Commands are the most basic feature you can add to your module. In order to add them, you need to implement two methods in your class: \textit{registerCommands} and \textit{handleCommand}.
\textit{registerCommands} should return the commands you want to process as an array of strings.
\begin{code}{php}
public function registerCommands() {
  return array("helloworld");
}
\end{code}
In the code example we have registered the command 'helloworld'. We now need to handle that command by implementing the 'handleCommand' function:
\begin{code}{php}
public function handleCommand(\$command, \$argument) {
  return "\<h1\>Hello, World!\</h1\>";
}
\end{code}
Now, when the user types \backslashhelloworld in their source, it will compile to \<h1\>Hello, World!\</h1\>! But, if the user types \backslashhelloworld\{InvalidArgument\}, it will also
compile to the same thing! It will just ignore the argument, but we might want to give the user a warning about that. For that, we make a call to the MarTeX object, which we can
access because we extended the MarTeXmodule class:
\begin{code}{php}
public function handleCommand(\$command, \$argument) {
  if (is_array(\$argument) or \$argument != "")) {
    \$this-\>MarTeX-\>parseError("(MarTeX/Helloworld) Warning: more arguments supplied to helloworld then required.");
  }
  return "\<h1\>Hello, World!\</h1\>";
}
\end{code}
You might notice that we first check if the argument is an array. That is because argument is an array if more then one argument is supplied, and a string when it is only a single argument.




