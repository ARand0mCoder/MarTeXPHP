\begin{paragraph}
    The figure module exposes commands that are designed to mimick the LaTeX package graphicx, with a few exceptions. 
    Maybe the most notable exception is that most of these commands only work inside a figure environment, and will result in "command not found" errors if you use them elsewhere.
\end{paragraph}

\subsubsection{Implemented commands}
\begin{itemize}
    \item{\descriptor{figure}{\command{includegraphics}{path/to/image}}{Use to display an image.}}
    \item{\descriptor{figure}{\command{caption}{caption text}}{Use to display text under your image.}}
    \item{\descriptor{figure}{\command{label}{label text}}{Use to give your image a label, 
                                            so you can use \textit{\command{ref}{label}} to reference it.}}
    \item{\descriptor{figure}{\command{width}{Width}}{ Set the width of your image. 
                    Width can have the following formats: 10px, 10\%, 10cm. }}
    \item{\descriptor{figure}{\command{height}{Height}}{ Set the height of your image. 
                    Height can have the following formats: 10px, 10\%, 10cm. }}
    \item{\descriptor{figure}{\command{alttext}{alt text}}{Set the alt text, also called hovertext, of your image.}}
\end{itemize}

\subsubsection{Settings}
\begin{paragraph}
    Reference labels are used to point to your image from the text. They put the following text in your caption: Figure N, where N is replaced with the number of your image. However, you might want to use a different text. For this you can access the global variable 'figureheader'. For example, you could do: \command{define}{figureheader}{image}. Now, the caption wil read: Image N. You should put the \command{define} above all figure environments.
\end{paragraph}

\subsubsection{Example}
\begin{code}{latex}
    \command{define}{figureheader}{image}
    \command{begin}{figure}
        \command{includegraphics}{test.png}
        \command{caption}{This is a test image}
        \command{label}{testimage}
        \command{width}{120px}
        \command{height}{30\%}
        \command{alttext}{A mouse!}
    \command{end}{figure} 
\end{code}

\begin{figure}
    \includegraphics{http://www.masswerk.at/bookmarklets/zapembeds/testimage.gif}
    \caption{This is a test image}
    \inline
    \label{testimage}
    \width{120px}
    \height{300}
    \alttext{A mouse!}
\end{figure} 
